\documentclass{article}
\usepackage[table]{xcolor}
\pagecolor{brown!4}
%\usepackage{graphicx}
\usepackage{fontspec}
\setmainfont{Noto Serif}
\usepackage{abstract}
%%%\usepackage{tabularray}


%\usepackage{luacode}
%\usepackage{etoolbox}
%\usepackage{multicol}
%\usepackage{xparse}

\usepackage[unicode,meta,
carian,
lycian,
]{xtransliteration}



%============================== Font
\newfontfamily\ftcafont{Noto Sans Carian}[Scale=1.5]%plain font
\newfontfamily\ftlcfont{Noto Sans Lycian}[Scale=1.5]%plain font


%\input{unicode_cb}
\newfontface\fcarian{NotoSansCarian-Regular.ttf}[Colour=blue]
\DeclareTextFontCommand\textcarian{\fcarian}
\newfontface\flycian{NotoSansLycian-Regular.ttf}[Colour=blue]
\DeclareTextFontCommand\textlycian{\flycian}


\newfontface\nrffontname{NotoSansCarian-Regular.ttf}

\title{The xtransliteration package}
\author{}
\date{}

\newcommand\tentry[5]{%
#1 &
\metac{#2} &
\metacb{#3} &
#4 &
#5{#4} \\

}

\newcommand\sep{\vspace{1.2ex}\hskip 0.3\textwidth\leaders\hrule width 0.3\textwidth\hskip 0.3\textwidth\vspace{1.8ex}}

\newcommand\dotexample[1]{%
\begin{quotation}\ttfamily\noindent #1 \end{quotation}%
}





\begin{document}
\maketitle
\begin{abstract}
The ~\texttt{xtransliteration} ~ package provides a set of \textsc{ASCII}-based input methods for various Unicode codeblock scripts, so that the scripts can be typeset using just the characters on an \textsc{ASCII} keyboard as input, for example Lycian \cdr{\lcts{abgd}}.
\par Scripts are selected by package option and by defining a corresponding font switch.
\par This document provides a very quick overview of the available scripts.

%\begin{center}
%\begin{tabular}{lllll}
%\rowcolor{\theadercolour}
%Script & Option & Font switch & Input & Output \\ 
%\hline
%\tentry{Carian}{carian}{ftcafont}{a p2 d l}{\cats}
%\tentry{Lycian}{lycian}{ftlcfont}{a e b bh}{\lcts}
%\hline
%\end{tabular}
%%%%\begin{tblr}{lllll}
%%%%%\rowcolor{\theadercolour}
%%%%Script & Option & Font switch & Input & Output \\ 
%%%%\hline
%%%%Carian & carian & ftcafont & a p2 d l & \textbackslash cats \\
%%%%%Lycian}{lycian}{ftlcfont}{a e b bh}{\lcts}
%%%%%\tentry{Carian}{carian}{ftcafont}{a p2 d l}{\cats}
%%%%%\tentry{Lycian}{lycian}{ftlcfont}{a e b bh}{\lcts}
%%%%\hline
%%%%\end{tblr}
%\end{center}
\end{abstract}

\sep




\section{Usage}
(a) \textsc{Activate a transliteration script}: -- Activate the package the usual way, and specify the desired script(s) as a comma-separated list of package options. For example, to switch on Carian and Lycian transliteration commands, do:

\begin{verbatim}
\usepackage[carian,lycian]{xtransliteration}
\end{verbatim}

\noindent (b) \textsc{Define a Font}: -- A font switch named \texttt{ft\textcolor{blue}{xx}font} is needed for the transliteration font, where \texttt{\textcolor{blue}{xx}} is the two-letter code for the script. For example, Carian and Lycian fonts are specified by:

\begin{verbatim}
\newfontfamily\ftcafont{Noto Sans Carian}
\newfontfamily\ftlcfont{Noto Sans Lycian}
\end{verbatim}

\newpage\tableofcontents

\sep

%**** Carian
\newpage\section{Carian}
\subsection{Available Commands \catagb }

{
\rowcolors{2}{blue!12!green!2}{yellow!25!green!12}
\begin{tabular}{lll}
\rowcolor{blue!12}
Type & Short Command &  General command \\
\hline
Codepoint & 
\cdr{\cauc{U+102A0}} & 
\cdr{\catrans[uc]{U+102A0}} \\
Unicode name & 
\cdr{\caun{a}} & 
\cdr{\catrans[un]{a}} \\
Typing shortcut & 
\cdr{\cats{a}} & 
\cdr{\catrans[ts]{a}} \\
Transliteration & 
\cdr{\cast{p2}} & 
\cdr{\catrans[st]{p2}} \\
\hline
Word & 
\cdr{\caw{kat}} & 
\cdr{\catext[w]{kat}} \\
Gloss & 
\scriptsize\cdr{\cagloss{kat}{dog}} & 
\scriptsize\cdr{\catext[gloss]{kat}{dog}} \\
Ruby & 
\cdr{\caruby{kat}} & 
\cdr{\catext[ruby]{kat}} \\
Ruby & 
\cdr{\caruby{k.a.t.}} & 
\cdr{\catext[ruby]{k.a.t.}} \\
\hline
\end{tabular}
}

\subsection{Example Transliteration \catagb }
\begin{quotation}
\cdr{\cats{kat}}

direct input: \textcarian{𐊨𐊣𐊠𐊦𐊹𐊸}

\cdr{\caruby{q.l.a.ld.i.ss.}}

\cdr{\caruby{qlaldiss. qlaldiss.}}

\cdr{\cagloss{ted}{father}}

\cdr{\cagloss{en}{mother}}
\end{quotation}


\textcarian{%
\doprmbx{66208}{66271}{'Carian'}{c:/windows/fonts/NotoSansCarian-Regular.ttf}}


%\catutorial

%\section{List of Unicode codepoints}\label{sec:listcauc}
%\begin{multicols}{4}\noindent
%\catag
%\cashowplainlistuc
%\eolist
%\end{multicols}



%**** Lycian
\newpage\section{Lycian}
\subsection{Available Commands \lctagb }

{
\rowcolors{2}{blue!12!green!2}{yellow!25!green!12}
\begin{tabular}{lll}
\rowcolor{blue!12}
Type & Short Command &  General command \\
\hline
Codepoint & 
\cdr{\lcuc{U+10280}} & 
\cdr{\lctrans[uc]{U+10280}} \\
Unicode name & 
\cdr{\lcun{a}} & 
\cdr{\lctrans[un]{a}} \\
Typing shortcut & 
\cdr{\lcts{a}} & 
\cdr{\lctrans[ts]{a}} \\
Transliteration & 
\cdr{\lcst{enx}} & 
\cdr{\lctrans[st]{enx}} \\
\hline
Word & 
\cdr{\lcw{ken}} & 
\cdr{\lctext[w]{ken}} \\
Gloss & 
\scriptsize\cdr{\lcgloss{kat}{dog}} & 
\scriptsize\cdr{\lctext[gloss]{kat}{dog}} \\
Ruby & 
\cdr{\lcruby{kat}} & 
\cdr{\lctext[ruby]{kat}} \\
Ruby & 
\cdr{\lcruby{k.a.t.}} & 
\cdr{\lctext[ruby]{k.a.t.}} \\
\hline
\end{tabular}
}


\subsection{Example Transliteration \lctagb }
\begin{quotation}
\cdr{\lcts{kat}}

\cdr{\lcgloss{esbe}{horse}}

\cdr{\lcgloss{trmm.mili}{the Lycian language}}
%\cdr{\caruby{q.l.a.ld.i.ss.}}
%
%\cdr{\cagloss{ted}{father}}
%
%\cdr{\cagloss{en}{mother}}
\end{quotation}


\textlycian{%
\doprmbx{66176}{66207}{'Lycian'}{c:/windows/fonts/NotoSansLycian-Regular.ttf}}


%\doprmb{67872}{67903}{'Lydian'}







\end{document}