\documentclass{article}
\usepackage{tcolorbox}
\tcbuselibrary{documentation}
\usepackage{minted}

%\tcbset{
%docexample/.style={colframe=ExampleFrame,colback=ExampleBack,
%before skip=\medskipamount,after skip=\medskipamount,
%fontlower=\footnotesize}
%}

\hypersetup{colorlinks=true}%,allcolors=violet}

\tcbset{
docexample/.style={colframe=violet,
bicolor,
colback=blue!9,
colbacklower=blue!2,
%color hyperlink=green,
boxrule=0.5pt,
before skip=\medskipamount,
after skip=\medskipamount,
fontlower=\footnotesize}
}


%\tcbset{doc head command={interior style={fill,left color=red!20!white,
%right color=blue!20!white}}}

\tcbset{doc head command={interior style={fill,color=yellow!10}}}

\tcbset{color definition=blue,
}

\newcommand\sqbrackets[1]{{\ttfamily [#1]}}

%=========================
\usepackage{tikz}
\usepackage{tkz-euclide}
\tcbuselibrary{minted,skins,breakable,xparse}
\usetikzlibrary{positioning}

\newenvironment{testlist}
               {\list{}{\leftmargin2cm\rightmargin\leftmargin}%
                \item\relax\footnotesize}
               {\endlist}



\definecolor{corAzulTema}{RGB}{0,66,137}
\definecolor{corFundoCaixas}{RGB}{245,245,245}


\tcbset{lilcode/.style={
    listing engine=minted,
    minted options={fontsize=\small,linenos,numbersep=3mm,breaklines},
    colback=corFundoCaixas,
    colframe=corAzulTema!40,
    fonttitle=\bfseries,
    listing only,
    left=7mm,
    enhanced,
    breakable,
    drop fuzzy shadow,
    before skip=\baselineskip,
%    hyperref options = {allcolors=black},
    grow to left by=13mm,grow to right by=15mm,
    overlay={\begin{tcbclipinterior}\fill[corAzulTema!40](frame.south west)rectangle([xshift=7mm]frame.north west);\end{tcbclipinterior}}
}}

\newtcbinputlisting[auto counter]{\latexcodefile}[3][]{%
    listing file={#3},
    title={\sffamily\large Listing \thetcbcounter:} \texttt{#2},
    colframe=corAzulTema,
    minted language=latex,
}

%                                       v = verbatim argument
%                                       s = optional star
%                                           \BooleanTrue or \BooleanFalse
\DeclareTotalTCBox{\lilcode}{ s v }{
	lilcode,
    reset,
    listing only,
    on line,
    enhanced,
    nobeforeafter,
    tcbox raise base,
    boxrule=0.7pt,
    top=1mm,
    bottom=0mm,
    right=1mm,
    left=1mm,
    boxsep=0.5pt,
    before upper={\vphantom{dlg}},
    colback=corFundoCaixas,
    colframe=corAzulTema,
    drop fuzzy shadow,
    arc=3pt}
{\mintinline{latex}{#2}}%

%=====
\ExplSyntaxOn
\tl_new:N \l_my_tl
\tl_new:N \l_myts_tl

\NewDocumentCommand{\cdr}{ m }{%
	\tl_set:Nn \l_my_tl { #1 }
	{ \ttfamily\color{blue}
  \detokenize{#1}
  }
  
  \enspace $\mapsto$ \enspace
  
  \tl_use:N \l_my_tl
}%

\ExplSyntaxOff



\begin{document}


\begin{docCommands}[
doc no index, % no index entries for this example
doc name = bfseries,
doc description=font,
] {}
Font switch to set the following text in bold face.
\begin{dispExample}
  text \bfseries bold text \mdseries normal text again
\end{dispExample}
\end{docCommands}

\begin{docCommands}[
doc no index, % no index entries for this example
doc name = itshape,
doc description=font,
] {}
Font switch to set the following text in italics.
\begin{dispExample}
  text \itshape italic text \upshape normal text again
\end{dispExample}
\end{docCommands}

\begin{docCommands}[
doc no index, % no index entries for this example
doc name = scshape,
doc description=font,
] {}
Font switch to set the following text in small caps.
\begin{dispExample}
  text \scshape Small Capitals \upshape normal text again
\end{dispExample}
\end{docCommands}


\begin{docCommands}[
doc no index, % no index entries for this example
doc name = slshape,
doc description=font,
] {}
Font switch to set the following text as slanted.
\begin{dispExample}
  \sffamily text \slshape slanted text \upshape normal text again
\end{dispExample}
\end{docCommands}


\begin{docCommands}[
doc no index, % no index entries for this example
doc name = upshape,
doc description=font,
] {}
Font switch to set the following text to upright.%, non-smallcaps.
\begin{dispExample}
    \itshape italic \scshape text \upshape upright text 
\end{dispExample}
\end{docCommands}

\begin{docCommands}[
doc no index, % no index entries for this example
doc name = ulcshape,
doc description=font,
] {}
Font switch to undo small caps.%, non-smallcaps.
\begin{dispExample}
    \itshape italic \scshape text \upshape upright \ulcshape text 
\end{dispExample}
\end{docCommands}



\begin{docCommands}[
doc no index, % no index entries for this example
doc name = textbf,
doc parameter = \marg{text},
doc description=font,
] {}
Font command to set \meta{text} in bold face.
\begin{dispExample}
  text \textbf{bold text} normal text again
\end{dispExample}
\end{docCommands}

\begin{docCommands}[
doc no index, % no index entries for this example
doc name = textit,
doc parameter = \marg{text},
doc description=font,
] {}
Font command to set \meta{text} as italic.
\begin{dispExample}
  text \textit{Felis gattus} normal text again
\end{dispExample}
\end{docCommands}
If the font does not have italics, slanted is used.


\begin{docCommands}[
doc no index, % no index entries for this example
doc name = textsl,
doc parameter = \marg{text},
doc description=font,
] {}
Font command to set \meta{text} as slanted.
\begin{dispExample}
  \sffamily text \textsl{Felis gattus} normal text again
\end{dispExample}
\end{docCommands}
If the font does not have a slanted face, italics are used.

\begin{docCommands}[
doc no index, % no index entries for this example
doc name = textulc,
doc parameter = \marg{text},
doc description=font,
] {}
Font command to set \meta{text} as non-small caps.
\begin{dispExample}
  \scshape text \textulc{non-small caps} and SC text again
\end{dispExample}
\end{docCommands}

%docCommand : index + \refCom
%docCommand* : no index entry

%\begin{docCommands}[
%doc no index, % no index entries for this example
%doc name = newtheorem,
%doc parameter = \marg{envname},
%]{
%{ },
%{ doc parameter = \marg{envname}\oarg{numbered within} },
%{ doc parameter = \oarg{numbered like}\marg{envname} },
%{ doc name = newtheorem* },
%}
%example
%\end{docCommands}


%\begin{docCommands}[
%doc no index, % no index entries for this example
%doc name = bfseries,
%] {
%		{ },
%		{ doc name = mdseries },
%		{ doc name = itshape },
%		{ doc name = scshape },
%		{ doc name = upshape },		
%}
%Font switches to set the weight as bold face (bf), medium weight (md); the shape as italic (it), small caps (sc), or upright (up).
%\end{docCommands}

%\begin{docEnvironment}[hoptionsi]{hnamei}{hparametersi}
%henvironment descriptioni
%\end{docEnvironment}

%docEnvironment : index + \refEnv{}
%docEnvironment* : no index entry


%docEnvironments

%\begin{docEnvironments}[
%doc no index, % no index entries for this example
%doc parameter = \oarg{options}\marg{title},
%doclang/environment content = box content,
%]{
%{
%doc name = redbox,
%doc description = a red colored box,
%},
%{
%doc name = greenbox,
%doc description = a green colored box,
%},
%{
%doc name = bluebox,
%doc description = a blue colored box,
%},
%{
%doc name = custombox,
%doc parameter = \oarg{options}\marg{color}\marg{title},
%doc description = a colored box,
%},
%}
%example
%\end{docEnvironments}

%docKey : refKey
%docKey*
%docKeys
%
%\begin{docKeys}[
%doc no index, % no index entries for this example
%doc keypath = mykeyroot,
%doc parameter = {=\meta{length}},
%]{
%{
%doc name = width,
%doc description = initially \texttt{10cm},
%},
%{
%doc name = height,
%doc description = initially \texttt{7cm},
%},
%}
%example
%\end{docKeys}


%%%docValue
%%%docValue*

%%%Counter
%%%Length
%%%Color

%%%cs
%%%meta
%%%marg
%%%oarg
%%%brackets



%\begin{docCommand}[hoptionsi]{hnamei}{hparametersi}
%hcommand descriptioni
%\end{docCommand}




\begin{docCommand}[
doc no index, % no index entries for this example
doc name = cdr,
doc parameter = \marg{code},
doc description=meta,
] {}%name
{}%parameters
Metacommand to print \meta{code} and then run it.
\begin{dispExample}
  The \cs{colorbox} command takes two parameters:\par
  text \cdr{\colorbox{red!12!yellow!80}{text}} text
\end{dispExample}
\end{docCommand}




\begin{docCommand}[
doc no index, % no index entries for this example
doc name = cs,
doc parameter = \marg{control sequence name},
doc description=meta,
] {}%name
{}%parameters
Metacommand to pre-pend the escape character (\textbackslash) and print \meta{control sequence name}.
\begin{dispExample}
  The command \cs{textit}\marg{text} will set \meta{text} 
  in italics.
\end{dispExample}
\end{docCommand}


\begin{docCommand}[
doc no index, % no index entries for this example
doc name = marg,
doc parameter = \marg{argument},
doc description=meta,
] {}%name
{}%parameters
Metacommand to print the mandatory \meta{argument} of a command in braces. See \refCom{oarg}.
\begin{dispExample}
  The command \cs{foo}\marg{arg1}\marg{arg2} will process 
  \meta{arg1} and \meta{arg2} \ldots
\end{dispExample}
\end{docCommand}

\begin{docCommand}[
doc no index, % no index entries for this example
doc name = oarg,
doc parameter = \oarg{argument},
doc description=meta,
] {}%name
{}%parameters
Metacommand to print the optional \meta{argument} of a command in brackets. See \refCom{marg}.
\begin{dispExample}
  \cs{newcommand}\{\cs{foo}\}\sqbrackets{2}\oarg{default}
  \marg{code} will use \meta{default} as \#1 if no optional 
  argument \meta{oarg} is specified: \cs{foo}\oarg{oarg}
  \marg{marg} \textit{versus} \cs{foo}\marg{marg}.
\end{dispExample}
\end{docCommand}


\begin{docCommand}[
doc no index, % no index entries for this example
doc name = brackets,
doc parameter = \marg{argument},
doc description=meta,
] {}%name
{}%parameters
Metacommand to print its \meta{argument} inside curly brackets (``braces''). The \meta{argument} may be empty. See \refCom{sqbrackets}.
\begin{dispExample}
  \cs{foo}\brackets{text} \cs{foo}\brackets{} \cs{foo}\marg{}
\end{dispExample}
\end{docCommand}


\begin{docCommand}[
doc no index, % no index entries for this example
doc name = sqbrackets,
doc parameter = \marg{argument},
doc description=meta,
] {}%name
{}%parameters
Metacommand to print its \meta{argument} inside square brackets (``brackets''). The \meta{argument} may be empty. See \refCom{brackets}.
\begin{dispExample}
  \cs{foo}\sqbrackets{text} \cs{foo}\sqbrackets{} \cs{foo}\oarg{}
\end{dispExample}
\end{docCommand}

%%%dispExample : docexample style
%%%dispListing : docexample style

\begin{docEnvironment}[
doc no index, % no index entries for this example
%doc name = dispListing,
%doc parameter = \marg{argument},
doc description=meta,
] {dispListing}%name
{}%parameters
Environment to print its contents as a code listing. See \refEnv{dispExample}.
\begin{dispExample}
  \begin{dispListing}
  Some text and \cs{foo}\oarg{arg1}\marg{arg2}
  \end{dispListing}
\end{dispExample}
\end{docEnvironment}




\begin{docEnvironment}[
doc no index, % no index entries for this example
%doc name = dispExample,
%doc parameter = \marg{argument},
doc description=meta,
] {dispExample}%name
{}%parameters
Environment to print its contents as a code listing and as resulting output. See \refEnv{dispListing}.
\begin{dispListing}
\begin{dispExample}
  Some text, \textit{italic} \textsc{small caps}
\end{dispExample}
\end{dispListing}
produces:  
  \begin{dispExample}
  Some text, \textit{italic} \textsc{small caps}
	\end{dispExample}
\end{docEnvironment}



%%%asquote = abstract as quotation

%%%colDef
%%%colOpt
%%%colFade

%%%tcbdocmarginnote

%%%tcbdocnew
%%%tcbdocupdated

The \lilcode{\textit{}} command has a matching switch-pair: an on-switch \lilcode{\itshape} (italic shape) and an off-switch \lilcode{\upshape} (upright shape).

\latexcodefile{Example listing}{listdcoarg.tex}

%\begin{testlist}
%qwerty
%
%qwerty
%\end{testlist}

\end{document}