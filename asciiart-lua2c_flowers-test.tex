%\documentclass[border=6pt]{standalone}
\documentclass{article}
\usepackage{luacode}
\usepackage{xcolor}
\usepackage{graphicx}
\usepackage{fontspec}
\setmonofont{FreeMono}

\directlua {require "asciiart3"}
\newcommand\bwasciitextshape[5]{\directlua{bwasciitextshape("#1",#2,#3,#4,"#5")}}
\newcommand\bwasciitextshade[5]{\directlua{bwasciitextshade("#1",#2,#3,#4,"#5")}}
\newcommand\bwascii[1]{\directlua{bwascii("#1")}}
\newcommand\colorascii[1]{\directlua{colorascii("#1")}}
\newcommand\bwframe[1]{\vspace*{\fill}\bwascii{#1}\vspace*{\fill}\newpage}
\newcommand\colorframe[1]{\vspace*{\fill}\colorascii{#1}\vspace*{\fill}\newpage}

\begin{document}
\ttfamily\frenchspacing
\newlength{\correctem}\settowidth{\correctem}{M}%to set the size of the minipage
\newlength{\correctex}\settowidth{\correctex}{x}%to set the line height
\pagestyle{empty}\centering

%\colorframe{lenna_128x128.ppm}
%\colorframe{lisa_150x224.ppm}
%\colorframe{knuth_192x227.ppm}
%\bwframe{einstein_150x206.pgm}
%\bwframe{sfmjk.pgm}
%%\bwascii{sfmjk.pgm}
%%\newpage
%\bwascii{ps.pgm}
\fbox{%
\bwasciitextshade{flower1.pgm}{1}{0}{256}{A flower, sometimes known as a bloom or blossom, is the reproductive structure found in flowering plants (plants of the division Magnoliophyta, also called angiosperms). The biological function of a flower is to facilitate reproduction, usually by providing a mechanism for the union of sperm with eggs. Flowers may facilitate outcrossing (fusion of sperm and eggs from different individuals in a population) resulting from cross pollination or allow selfing (fusion of sperm and egg from the same flower) when self-pollination occurs. }
}
\fbox{%
\bwasciitextshade{flower1.pgm}{1}{170}{256}{A flower, sometimes known as a bloom or blossom, is the reproductive structure found in flowering plants (plants of the division Magnoliophyta, also called angiosperms). The biological function of a flower is to facilitate reproduction, usually by providing a mechanism for the union of sperm with eggs. Flowers may facilitate outcrossing (fusion of sperm and eggs from different individuals in a population) resulting from cross pollination or allow selfing (fusion of sperm and egg from the same flower) when self-pollination occurs. }
}
\newpage
\fbox{%
\bwasciitextshape{flower1.pgm}{1}{170}{256}{A flower, sometimes known as a bloom or blossom, is the reproductive structure found in flowering plants (plants of the division Magnoliophyta, also called angiosperms). The biological function of a flower is to facilitate reproduction, usually by providing a mechanism for the union of sperm with eggs. Flowers may facilitate outcrossing (fusion of sperm and eggs from different individuals in a population) resulting from cross pollination or allow selfing (fusion of sperm and egg from the same flower) when self-pollination occurs. }
}
\newpage
\fbox{%
\bwasciitextshape{flower1.pgm}{1}{0}{170}{A flower, sometimes known as a bloom or blossom, is the reproductive structure found in flowering plants (plants of the division Magnoliophyta, also called angiosperms). The biological function of a flower is to facilitate reproduction, usually by providing a mechanism for the union of sperm with eggs. Flowers may facilitate outcrossing (fusion of sperm and eggs from different individuals in a population) resulting from cross pollination or allow selfing (fusion of sperm and egg from the same flower) when self-pollination occurs. }
}
\newpage
\fbox{%
\bwascii{flower1.pgm}
}

\end{document}

convert -compress none -resize ___OutputWidth___ ___InputFile___ ___output___.ppm
convert -compress none -resize ___OutputWidth___ ___InputFile___ ___output___.pgm